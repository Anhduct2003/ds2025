\documentclass[a4paper,12pt]{article}
\usepackage{fontspec} % For custom fonts
\usepackage{listings}

% Set Arial as the default font
\setmainfont{Arial}

\title{Lab 3 - MPI File Transfer}
\author{BA12-051 - Tran Anh Duc}

\begin{document}

\maketitle

\section{Protocol}
The server and client exchange information using the MPI (Message Passing Interface) communication standard. The protocol includes these steps:
\begin{itemize}
    \item The client reads the file content and sends it to the server using an MPI message.
    \item The server receives the file content and stores it in a specified location.
    \item Once the transfer is complete, the server sends an acknowledgment to the client.
\end{itemize}

\section{System Architecture}
The system consists of two entities — the client and the server — implemented as MPI processes. The client and server communicate through MPI messages:
\begin{itemize}
    \item \textbf{Client (Rank 1)}: Reads the file and sends the file data to the server.
    \item \textbf{Server (Rank 0)}: Receives the file data and writes it to a local file.
\end{itemize}
MPI enables efficient communication between the client and server processes.

\section{File Transfer}
The file transfer mechanism involves sending and receiving MPI messages containing file data. Below is the implementation for the server and client.

\subsection{Server Code}
The server receives the file data from the client using \texttt{MPI\_Recv} and writes it to a file. Here is the implementation:

\begin{lstlisting}[language=Python]
from mpi4py import MPI

def start_server(file_name="received_file.txt"):
    comm = MPI.COMM_WORLD
    rank = comm.Get_rank()

    if rank == 0:  # Server role
        print("Server waiting for file data...")
        file_data = comm.recv(source=1, tag=11)  # Receive file data from client
        print("File data received. Writing to file...")

        with open(file_name, "wb") as file:
            file.write(file_data)
        
        print(f"File '{file_name}' received successfully!")

if __name__ == "__main__":
    start_server()
\end{lstlisting}

\subsection{Client Code}
The client reads the file content and sends it to the server using \texttt{MPI\_Send}. Here is the implementation:

\begin{lstlisting}[language=Python]
from mpi4py import MPI

def send_file(file_path):
    comm = MPI.COMM_WORLD
    rank = comm.Get_rank()

    if rank == 1:  # Client role
        print("Client reading file and sending data...")
        try:
            with open(file_path, "rb") as file:
                file_data = file.read()

            comm.send(file_data, dest=0, tag=11)  # Send file data to server
            print("File sent successfully!")
        except FileNotFoundError:
            print(f"File '{file_path}' not found. Exiting.")

if __name__ == "__main__":
    send_file("transferfile.txt")
\end{lstlisting}

\section{Advantages of MPI Transfer}
\begin{itemize}
    \item \textbf{Scalability}: MPI supports distributed systems with multiple processes, making it ideal for larger networks.
    \item \textbf{Efficiency}: Communication between processes is optimized for performance.
    \item \textbf{Flexibility}: MPI allows custom communication patterns, such as point-to-point or collective communication.
\end{itemize}

\section{Comparison with TCP Socket Transfer}
\begin{table}[h!]
\centering
\begin{tabular}{|l|l|l|}
\hline
\textbf{Aspect} & \textbf{MPI Transfer} & \textbf{TCP Socket Transfer} \\
\hline
Ease of Implementation & Moderate & Simple \\
\hline
Scalability & High & Limited \\
\hline
Performance & High (Optimized for HPC) & Moderate \\
\hline
Cross-Platform Compatibility & High & High \\
\hline
\end{tabular}
\caption{Comparison of MPI and TCP Socket Transfer}
\end{table}

\end{document}
